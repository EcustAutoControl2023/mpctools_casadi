\documentclass[twocolumn]{article}

\usepackage[left=1in,right=1in,top=1in,bottom=1in]{geometry}
\usepackage{mathtools,bm,textcomp}
\usepackage[dvipsnames]{xcolor}

% Stuff for source codes.
\usepackage{listings}
\providecommand{\lstinline}{}
\lstdefinestyle{python}{
    language=Python,
    basicstyle=\ttfamily\footnotesize,
    keywordstyle=\color{Blue}\ttfamily,
    stringstyle=\color{Purple}\ttfamily,
    commentstyle=\color{ForestGreen}\ttfamily,
    upquote=true,
    showstringspaces=false,
%    numbers=left,
    stepnumber=10,
    firstnumber=0,
    numberstyle=\scriptsize,
    frame=L,
    escapechar=@,
    morekeywords={as},
}
\lstset{style=python}

\newcommand{\smallurl}[1]{\texttt{\scriptsize$<$#1$>$}}
\newcommand{\funcname}[1]{\vspace{.25em}\noindent\texttt{#1}\vspace{.25em}}
\newcommand{\casadi}{CasADi}

\title{\texttt{mpc-tools-casadi} Cheat Sheet}

\begin{document}

\pagestyle{empty}
\maketitle

\thispagestyle{empty}

\paragraph*{Obtaining \texttt{mpc-tools-casadi}}

The latest files can be found on \smallurl{https://hg.cae.wisc.edu/hg/mpc-tools-casadi}.

\paragraph*{Getting Started}

Functions are split between two modules: \texttt{mpc\_tools\_casadi.py} and \texttt{colloc.py}.
Typically, you will only need to import the former, e.g.
%
\begin{lstlisting}[style=python]
import mpc_tools_casadi as mpc
\end{lstlisting}
%
You will need to have \casadi{} installed as well.
You shouldn't need to use any functions whose names begin with two underscores (e.g. \lstinline@__getCasadiSymbols@). 

Many functions will have optional arguments or default values that aren't listed below.
Consult the \texttt{mpc\_tools\_casadi} module to see what options are available.

\paragraph*{Simulating Nonlinear Systems}

To facilitate nonlinear simulations, we provide the \texttt{OneStepSimulator} class, which is essentially a wrapper of \casadi's \texttt{Integrator} object.
To initialize, the syntax is
%
\begin{lstlisting}
model = OneStepSimulator(ode,Delta,Nx,Nu)
\end{lstlisting}
%
Where \texttt{ode} is a Python function that takes two keyword arguments \texttt{x} and \texttt{u} of lengths \texttt{Nx} and \texttt{Nu}.
Optional arguments \texttt{Nd} or \texttt{Nw} can be set to positive integers to add additional arguments.

Once the object has been build, one timestep can be simulated using
\begin{lstlisting}
xnext = model.sim(x,u)
\end{lstlisting}

\paragraph*{Building \casadi{} Functions}

To simplify building \casadi{} function objects, we have a few convenience wrappers.

\funcname{getCasadiFunc(f,Nx,Nu,Nd)}

Takes a Python function and sizes of $x$, $u$, and/or $d$ to build a \casadi{} \texttt{MXFunction} object.
Note that the original function \texttt{f} should return a list of values (or at least something that can be passed as an argument to \casadi's \texttt{vertcat}).

\funcname{getCasadiFuncGeneralArgs(f,varsizes)}

For functions with arguments other than $x$, $u$, or $d$, use this version instead. \texttt{varsizes} is a list of integers giving the sizes of the inputs to \texttt{f} in order.

\funcname{getCasadiIntegrator(f,Delta,Nx,Nu,Nd)}

Returns an \texttt{Integrator} object to integrate the Python function \texttt{f} from time 0 to \texttt{Delta}.

\funcname{getRungeKutta4(f,Delta,M=1)}

Returns an explicit Runge-Kutta 4th order discretization with \texttt{M} steps of size \texttt{Delta}/\texttt{M}.
Note that for this function, \texttt{f} must already be an \texttt{MXFunction}, i.e. you should use it AFTER calling \texttt{getCasadiFunc}.

\paragraph*{Solving MPC Problems}





\end{document}