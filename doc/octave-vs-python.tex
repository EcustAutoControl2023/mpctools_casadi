\documentclass{article}

\usepackage[left=.25in,right=.25in,top=.75in,bottom=.75in]{geometry}
\usepackage{mathtools,bm,textcomp,multicol,parcolumns,enumitem,array}
\usepackage[dvipsnames]{xcolor}
\usepackage[T1]{fontenc}
\usepackage{textcomp}

% Stuff for source codes.
\usepackage{listings}
\providecommand{\lstinline}{}
\lstdefinestyle{python}{
    language=Python,
    basicstyle=\ttfamily\footnotesize,
    keywordstyle=\color{Blue}\ttfamily,
    stringstyle=\color{Purple}\ttfamily,
    commentstyle=\color{ForestGreen}\ttfamily,
    upquote=true,
    showstringspaces=false,
%    numbers=left,
    stepnumber=10,
    firstnumber=0,
    numberstyle=\scriptsize,
    escapechar=@,
    morekeywords={as},
    upquote=true,
}

\lstdefinestyle{matlab}{
    language=Matlab,
    basicstyle=\ttfamily\footnotesize,
    keywordstyle=\color{Blue},
    stringstyle=\color{Purple},
    commentstyle=\color{ForestGreen},
    upquote=true,
    morekeywords={endfunction},
}

\newcommand{\smallurl}[2][\scriptsize]{\texttt{#1$<$#2$>$}}
\newcommand{\funcname}[2][.25em]{\vspace{#1}\noindent\texttt{#2}\nopagebreak\vspace{#1}}
\newcommand{\casadi}{CasADi}

%\newcommand{\sepline}{\hrule}
\providecommand{\sepline}{\vspace{-2em}}

\title{Octave vs. Python for Example 1.11}

\begin{document}

%\pagestyle{empty}
%\thispagestyle{empty}

\begin{center}
    \LARGE Octave vs. Python for Example 1.11
\end{center}

Below, we present an example file to show that, for our purposes, Python isn't that much different from Octave/\textsc{Matlab}
On the left side, we show the the script written using Octave (\textsc{Matlab} compatibility requires breaking out the subfunctions), while on the right, we show the script rewritten to use Python+Casadi (with a bit of \texttt{mpc-tools-casadi} as well).

\hspace{1em}

\input{sidebyside-cstr.tex}

\end{document}
