\documentclass[xcolor=dvipsnames]{beamer}

\usepackage{listings}
\providecommand{\lstinline}{}

\lstdefinestyle{makefile}{
    language={[gnu]make},
    numbers=left,
    frame=lines,
    basicstyle=\ttfamily,
    keywordstyle=\color{Blue}\ttfamily,
    stringstyle=\color{Purple}\ttfamily,
    commentstyle=\color{ForestGreen}\ttfamily,
    escapechar=@
}

\lstdefinestyle{shell}{
    language=bash,
    basicstyle=\ttfamily\footnotesize,
}

\lstdefinestyle{python}{
    language=Python,
    basicstyle=\ttfamily,
    keywordstyle=\color{Blue}\ttfamily,
    stringstyle=\color{Purple}\ttfamily,
    commentstyle=\color{ForestGreen}\ttfamily,
    escapechar=@,
}

\newcommand{\codeemph}[1]{\textcolor{NavyBlue}{\small\bfseries #1}}

\newcommand{\smallurl}[2][\footnotesize]{\texttt{#1 #2}}

\title{MPC with Casadi/Python}
\date{March 11, 2015}
\author{Michael Risbeck}

\usepackage{lmodern}
%\renewcommand{\ttdefault}{pcr} % Use different tt font.

\begin{document}

\frame{\titlepage}


\begin{frame}[fragile]{Ubuntu/Debian Installation}

NumPy, Matplotlib, Spyder
\begin{itemize}
    \item Standard package: \lstinline[style=shell]@$ sudo apt-get spyder@
\end{itemize}

\medskip

CasADi:
\begin{itemize}
    \item Check dependencies from \smallurl{https://github.com/casadi/casadi/wiki/Binaryinstallationlinux}
    \item Download from \smallurl{http://sourceforge.net/projects/casadi/files/CasADi/}
    \item Install: \lstinline[style=shell]@$ dpkg -i <filename>.deb@
\end{itemize}

\medskip

Our python modules
\begin{itemize}
    \item Clone Mercurial repo: \lstinline[style=shell]!$ hg clone https://hg.cae.wisc.edu/hg/mpc-tools-casadi!
\end{itemize}

\end{frame}

\begin{frame}[fragile]{Developers}

A few extra steps if you want to be able to push changes:

\begin{itemize}
    \item Get Jim to authorize you
    \item Go to \smallurl{https://my.cae.wisc.edu/}
    \item Web Tools $\rightarrow$ Repositories
    \item Set Alternate Repository Passwords
    \item Add lines to \smallurl{.hg/hgrc}:
\end{itemize}

\begin{lstlisting}[style=shell]
    [auth]
    .prefix = hg.cae.wisc.edu/hg
    .username = <username>
    .password = <password>
\end{lstlisting}

\end{frame}

\begin{frame}{Windows Installation}

A Windows Python distribution.
\begin{itemize}
    \item E.g., Python(x,y): \smallurl{https://code.google.com/p/pythonxy/}
\end{itemize}

\medskip

CasADi:
\begin{itemize}
    \item Check dependencies from \smallurl{https://github.com/casadi/casadi/wiki/BinaryInstallationWindows}
    \item Download from \smallurl{http://sourceforge.net/projects/casadi/files/CasADi/}
    \item Install as you would normal Windows programs
\end{itemize}

\medskip

Our python modules
\begin{itemize}
    \item Download Mercurial repo: \smallurl{https://hg.cae.wisc.edu/hg/mpc-tools-casadi}
    \item Download \texttt{.zip} file on the left.
\end{itemize}

\end{frame}

\begin{frame}{Why did we need to write this code?}

\begin{itemize}
    \item We plan to solve nonlinear MPC problems.
    \item CasADi is more robust than our \texttt{mpc-tools}
    \item However, setting up an MPC problem in CasADi takes a fair bit of code
    \item Everyone copy/pasting their own code is bad.
    \item A simpler interface means we (and others) can save a lot of time.
\end{itemize}

\end{frame}

\begin{frame}[fragile]{From official CasADi Examples}

\begin{columns}
    \begin{column}{.75\textwidth}
        
\begin{lstlisting}[style=python,basicstyle=\ttfamily\fontsize{6}{8}\selectfont]
# For all collocation points: eq 10.4 or 10.17 in Biegler's book
# Construct Lagrange polynomials to get the polynomial basis at
# the collocation point
for j in range(deg+1):
    L = 1
    for j2 in range(deg+1):
        if j2 != j:
            L *= (tau-tau_root[j2])/(tau_root[j]-tau_root[j2])
    
    lfcn = SXFunction([tau],[L])
    lfcn.init()
    # Evaluate the polynomial at the final time to get the
    # coefficients of the continuity equation
    lfcn.setInput(1.0)
    lfcn.evaluate()
    D[j] = lfcn.getOutput()
    
    # Evaluate the time derivative of the polynomial at all
    #collocation points to get the coefficients of the
    #continuity equation
    tfcn = lfcn.tangent()
    tfcn.init()
    for j2 in range(deg+1):
        tfcn.setInput(tau_root[j2])
        tfcn.evaluate()
        C[j][j2] = tfcn.getOutput()
\end{lstlisting}
    \end{column}
    \begin{column}{.25\textwidth}
        We don't want everyone writing this themselves!
    \end{column}
\end{columns}

\end{frame}

\begin{frame}{What do we have so far?}
    \begin{itemize}
        \item Discrete-time linear MPC
        \item Discrete-time nonlinear MPC
        \begin{itemize}
            \item Explicit models
            \item Runge-Kutta discretization
            \item Collocation
        \end{itemize}
        \item Basic plotting function
        \item Example scripts
        \begin{itemize}
            \item Linear
            \item Solution of linear as nonlinear
            \item Periodic linear
            \item Example 2-8
            \item Simple collocation
        \end{itemize}
    \end{itemize}
    
\end{frame}

\begin{frame}[fragile,allowframebreaks]{Example Script}

Example script for a simple linear MPC problem.

\begin{lstlisting}[style=python,basicstyle=\ttfamily\fontsize{8}{10}\selectfont]
# MPC for a multivariable system.

# Imports.
import numpy as np
@\codeemph{import mpc\_tools\_casadi as mpc}@

# Define continuous time model.
Acont = np.array([[0,1],[0,-1]])
Bcont = np.array([[0],[10]])
n = Acont.shape[0] # Number of states.
p = Bcont.shape[1] # Number of control elements

# Discretize.
dt = .025
N = 200
t = np.arange(N+1)*dt
@\codeemph{(Adisc,Bdisc) = mpc.c2d(Acont,Bcont,dt)}@
A = [Adisc]
B = [Bdisc]

# Bounds on u.
umax = 1
ulb = [np.array([-umax])]
uub = [np.array([umax])]
bounds = dict(uub=uub,ulb=ulb)

# Define Q and R matrices and q penalty for periodic solution.
Q = [np.diag([1,0])]
q = [np.zeros((n,1))]
R = [np.eye(p)]

# Initial condition.
x0 = np.array([10,0])

# Solve linear MPC problem.
@\codeemph{solution = mpc.lmpc(A,B,x0,N,Q,R,q=q,bounds=bounds,verbosity=1)}@
x = solution["x"]
u = solution["u"]

# Plot things.
@\codeemph{fig = mpc.mpcplot(x,u,t,np.zeros(x.shape),xinds=[0])}@
fig.show()
\end{lstlisting}

\end{frame}

\begin{frame}{Output}
\begin{center}
    \includegraphics[height=.8\textheight]{mpcexample.pdf}
\end{center}
\end{frame}

\begin{frame}{What do we still need?}
    \begin{itemize}
        \item Continuous-time formulation
        \begin{itemize}
            \item Time-varying functions
            \item Quadrature for objective function
            \item DAE systems
        \end{itemize}
        \item Quality guess generation
        \begin{itemize}
            \item Solve sequence of smaller problems
            \item Use as initial guess for large problem
        \end{itemize}
        \item Moving horizon simulation
        \begin{itemize}
            \item Generate all constraints once
            \item Select appropriate subset for each step
            \item Use previous solution as initial guess
            \item OOP approach would be nice
        \end{itemize}
    \end{itemize}
\end{frame}

\begin{frame}{Additional Notes}

\begin{itemize}
    \item We may not be able to write good CasADi code, but we at least want to write good Python code
    \begin{itemize}
        \item \smallurl{http://docs.python-guide.org/en/latest/writing/style/}
        \item \smallurl{https://www.python.org/dev/peps/pep-0008/}
    \end{itemize}
    \item NumPy is verbose and takes getting used to, especially if you're using matrices
    \item Duck-typing makes me nervous, but it's kind of a major paradigm of Python
    \begin{itemize}
        \item Lists vs. NumPy vectors
        \item NumPy matrices vs. arrays
        \item CasADi \texttt{MX} vs. \texttt{SX}
    \end{itemize}
    \item ``Easier to Ask Forgiveness than Permission'' is challenging if you don't know what exceptions might be raised
    \begin{itemize}
        \item Takes some getting used to if you've been raised on ``Look Before You Leap'' (common in Octave/\textsc{Matlab})
        \item A brief discussion: \smallurl{http://code.activestate.com/lists/python-list/337643/}
    \end{itemize}
\end{itemize}

\end{frame}

\end{document}